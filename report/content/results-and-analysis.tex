%-----------------------------------------------------------------------------
% Author: Ramsey (Rayla) Phuc
% Alias: Rayla Kurosaki
% GitHub: https://github.com/rkp1503
%-----------------------------------------------------------------------------
\section{Results and Analysis}\label{sec:results-and-analysis}

\subsection{First set of initial conditions}\label{subsec:first-set-of-initial-conditions}

Consider the case with the following parameters and initial conditions:

\[
\begin{dcases}
    m_1 = 1\ \text{kg}\\
    m_2 = 1\ \text{kg}
\end{dcases},\ \begin{dcases}
    l_1 = 1\ \text{m}\\
    l_2 = 1\ \text{m}
\end{dcases},\ \begin{dcases}
    \theta_{1,0} = 45^\circ\\
    \theta_{2,0} = 0^\circ
\end{dcases},\ \begin{dcases}
    \omega_{1,0} = 0^\circ\text{/s}\\
    \omega_{2,0} = 0^\circ\text{/s}
\end{dcases},\ 0\leq t\leq10
\]

where each $\theta_{i,0}$ and $\omega_{i,0}$ are the initial starting angles and angular velocities for each bob. \myref[Figure]{fig:2} is a collection of graphs that will be generated at the end of the program. \myref[Figure]{fig:2c} and \myref[Figure]{fig:2d} are time plots for the position and angular velocity of each bob in the double pendulum system respectively. For a time of 10 seconds, it is clear that the double pendulum system under this set of initial conditions shows a consistent periodic behavior. These plots show that the double pendulum system has a period about 2.6383 seconds. \myref[Figure]{fig:2a} and \myref[Figure]{fig:2b} are phase plane plots for the position and angular velocity of each bob in the double pendulum system respectively. It is more apparent in the animation when running the code, but these phase plane plots show that there is a consistent pattern in how the system behaves. Finally, \myref[Figure]{fig:2e} is a time plot that plots the Potential, Kinetic, and Total energies against time. Here we can see that not only the consistent periodic behavior exist, but energy is conserved.

\subsection{Slightly changing one initial condition}\label{subsec:slightly-changing-one-initial-condition}
Lets use the same set-up but change one of the initial conditions slightly, which is $\theta_{2,0} = 1^\circ$ and rerun the program. \myref[Figure]{fig:3} is a set of graphs that are generated under this set of initial conditions. Compared to \myref[Figure]{fig:2c} and \myref[Figure]{fig:2d}, \myref[Figure]{fig:3c} and \myref[Figure]{fig:3d} look very similar. However, we can see that the bobs' position and angular velocity starts to differ after each period in \myref[Figure]{fig:3c} adn \myref[Figure]{fig:3d} respectively. The system does show a periodic behavior, but each period is slightly different from the previous period. This is more apparent when we look at \myref[Figure]{fig:3a} and \myref[Figure]{fig:3b}. These figures show that the system starts to be inconsistent while maintaining its periodic behavior over time. Just by increasing the initial angle of bob 2 by $1^\circ$, the system starts to diverge from its initial period. This shows that the Double Pendulum System is a chaotic system. \myref[Figure]{fig:3e} shows how the energy is conserved under these initial conditions. While energy is conserved, the amount of Potential and Kinetic energy differs as time goes on, which aligns with the fact that the system behaves in a chaotic way.
