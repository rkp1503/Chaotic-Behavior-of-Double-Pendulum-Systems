\section{Python code}\label{sec:python-code}
\lstset{style=mystyle}

\subsection{System of Ordinary Differential Equations}\label{subsec:system-of-ordinary-differential-equations}
\begin{lstlisting}[language=Python]
def df(F, t, params) -> np.array:
    # Unpack variables
    theta_1: float = F[0]
    theta_2: float = F[1]
    omega_1: float = F[2]
    omega_2: float = F[3]

    # Unpack parameters
    m_1: float = params[0]
    m_2: float = params[1]
    l_1: float = params[2]
    l_2: float = params[3]
    g: float = params[4]

    # System of differential equations
    theta_1_t: float = omega_1
    theta_2_t: float = omega_2

    # Breaking down each component since the expression is too big
    A1: float = -g * (2 * m_1 + m_2) * np.sin(theta_1)
    A2: float = - m_2 * g * np.sin(theta_1 - 2 * theta_2)
    A3: float = - 2 * m_2 * np.sin(theta_1 - theta_2)
    A4: float = (omega_2 ** 2) * l_2 + (omega_1 ** 2) * l_1 * np.cos(theta_1 - theta_2)
    B: float = l_1 * (2 * m_1 + m_2 - m_2 * np.cos(2 * (theta_1 - theta_2)))
    omega_1_t: float = (A1 + A2 + A3 * A4) / B

    # Breaking down each component since the expression is too big
    C1: float = 2 * np.sin(theta_1 - theta_2)
    C2: float = (omega_1 ** 2) * l_1 * (m_1 + m_2)
    C3: float = g * (m_1 + m_2) * np.cos(theta_1)
    C4: float = (omega_2 ** 2) * l_2 * m_2 * np.cos(theta_1 - theta_2)
    D: float = l_2 * (2 * m_1 + m_2 - m_2 * np.cos(2 * (theta_1 - theta_2)))
    omega_2_t: float = C1 * (C2 + C3 + C4) / D
    return np.array([theta_1_t, theta_2_t, omega_1_t, omega_2_t], dtype=float)
\end{lstlisting}

\subsection{My Implementation of the Runge-Kutta Order 4 Method}
\label{subsec:my-implementation-of-the-runge-kutta-order-4-method}
\begin{lstlisting}[language=Python]
def rk4(df, r, ts, h, params):
    rs = [r]
    for i in range(1, len(ts)):
        f1 = df(rs[i - 1], ts[i - 1], params)
        f2 = df(rs[i - 1] + (h / 2) * f1, ts[i - 1] + (h / 2), params)
        f3 = df(rs[i - 1] + (h / 2) * f2, ts[i - 1] + (h / 2), params)
        f4 = df(rs[i - 1] + h * f3, ts[i - 1] + h, params)
        r = rs[i - 1] + (h / 6) * (f1 + 2 * f2 + 2 * f3 + f4)
        rs.append(r)
        pass
    return get_data(rs)
\end{lstlisting}

\subsection{Helper Function}\label{subsec:get-data-helper-function}
\begin{lstlisting}[language=Python]
def get_data(data):
    rs, cs = len(data), len(data[0])
    temp = np.empty((cs, rs), dtype=float)
    for r in range(0, rs):
        for c in range(0, cs):
            temp[c][r] = data[r][c]
            pass
        pass
    if len(temp) == 1:
        return temp[0]
    return temp
\end{lstlisting}
